\documentclass{article}
\usepackage{graphicx}
\usepackage[a4paper, margin=1.5cm]{geometry}
\usepackage[colorlinks=true, allcolors=blue]{hyperref}
\usepackage{url}

\title{DT2470 Music Informatics Final Project}
\date{October 2025}

\author{
    \begin{minipage}[t]{0.24\textwidth}
        \centering
        Matei Cananau \\
        MSc Machine Learning \\
        \href{mailto:cananau@kth.se}{cananau@kth.se}
    \end{minipage}
    \hfill
    \begin{minipage}[t]{0.24\textwidth}
        \centering
        Arvid Ljung \\
        MSc Machine Learning \\
        \href{mailto:arvidlju@kth.se}{arvidlju@kth.se}
    \end{minipage}
    \hfill
    \begin{minipage}[t]{0.24\textwidth}
        \centering
        Matej Priesol \\
        MSc Machine Learning \\
        \href{mailto:priesol@kth.se}{priesol@kth.se}
    \end{minipage}
    \hfill
    \begin{minipage}[t]{0.24\textwidth}
        \centering
        Matei Cananau \\
        MSc Machine Learning \\
        \href{mailto:cananau@kth.se}{cananau@kth.se}
    \end{minipage}
}

\begin{document}

\maketitle

%% TABLE OF CONTENTS
\tableofcontents

(Around 4-5 pages long)
%% GRADING CRITERA [remove when done]

% \includegraphics[width=\textwidth]{figures/grading_criteria.png}

\newpage

%% INTRODUCTION
\section{Introduction}

Earlier this year, Apple Music introduced their new "AutoMix" feature, which automatically creates smooth transition mixes between songs in a playlist. \cite{apple2025}. No public information is available regarding the technical details of this feature, except that it uses Apple Intelligence to apply time stretching and beat matching techniques.

A few months later, Spotify launched their own audio mixing function, allowing users to manually create DJ-style transitions between tracks in their playlists from one tempo and key pairing into another \cite{spotify2025}.

This project attempts to achieve similar functionality by extracting features from audio tracks and using MIR techniques to create smooth transitions between songs.

%% FEATURE EXTRACTION
\section{Feature Extraction}

%% MODELING METHODS
\section{Modeling Methods}
\cite{vandeveire2018automateddj}

%% EVALUATION
\section{Evaluation}

%% CONCLUSION
\section{Conclusion}

%% REFERENCES
\bibliographystyle{plainurl}
\bibliography{references}

\end{document}